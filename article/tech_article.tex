\documentclass{article}

\usepackage{graphicx}
\usepackage{amsmath}
\usepackage{amssymb}
\usepackage{hyperref}

\title{My Technical Article}
\author{Ghanem Ghanem and Kumail Raza}
\date{\today}

\begin{document}

\maketitle

\section{Introduction}

Blockchain technology has shown great potential in the healthcare industry, 
particularly in the area of personal health record (PHR) management. 
By using a decentralized blockchain-based system, patients can have greater 
control over their health data, and healthcare providers can have more efficient 
and secure access to that data. In this article, we present our implementation of 
a PHR blockchain system and discuss our design choices, implementation challenges,
and limitations.

Our PHR blockchain system is designed to enable patients to manage their own health records on the blockchain, 
which can then be accessed and updated by authorized healthcare providers. The system is built using the Ethereum blockchain, 
which provides a decentralized and secure platform for storing and processing health data. 
We also use smart contracts to automate the management of health records and ensure that access to the data is restricted only to authorized users.

\section{PHR Implementation}
*** text here ***

\section{Limitations and Scalability of A PHR Blockchain System}

Scalability in the context of a blockchain is the capacity for a blockchain
to process transactions and store data. Scalability in blockchain has always
been a debated topic. A decentralized blockchain is immutable, all data that 
is added to the blockchain will stay on the blockchain and it cannot be changed
or deleted. You can imagine how scalability starts to become an issue. 
For example if our PHR blockchain could potentially have millions of users, 
each having multiple blocks on the chain (each block might represent a 
diagnosis, or a doctor visit). There have been many proposed solutions, 
one of the popular ones is to store data off-chain. This means that the 
blockchain will not store the data, instead a traditional database or an 
IFSA will store the encrypted data. In a paper by Dr. Andrew Fang discussed 
the limitations that a PHR blockchain system could have. He listed three 
examples, those being scalability, privacy and usability.

\subsection{Proposed Solutions}

There have been several proposed solutions to address the issue of scalability
in blockchain systems. One popular approach is to use sharding, which involves splitting
the blockchain into smaller parts called shards, each of which can process transactions independently. 
Another approach is to use off-chain solutions, such as the Lightning Network, 
to handle smaller transactions off the main blockchain.

\section{Conclusion}

In conclusion, scalability remains a key challenge for blockchain systems, 
and there is ongoing research and development to address this issue. 
By exploring different approaches and solutions, we can continue to advance 
the field and enable the widespread adoption of blockchain technology.

\end{document}
